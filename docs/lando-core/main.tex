\documentclass{article}
\usepackage[utf8]{inputenc}
\usepackage{geometry}
\geometry{
left=1.4in,
right=1.4in,
top=1in,
bottom=1in,
}

\usepackage[T1]{fontenc}
\usepackage{longtable}
\usepackage{textcomp}
\usepackage{array}
\usepackage{todonotes}
\usepackage{mathtools}

% Grammar definition helpers
\newenvironment{grammar}{%
  \global\def\grammarVertAdjustment{0mm}
  \newcommand{\gline}{\hline \\[-0.5em]}
  \newcommand{\gcont}{\\ & &}
  \newcommand{\gor}{\ensuremath{\mid}}
  \newcommand{\grule}[4][0mm]{##2 & ::= & ##3 & \mbox{\scriptsize $\textit{##4}$} \\[##1] }
  \newcommand{\gruleplain}[3][0mm]{##2 & \hphantom{::=} & & \mbox{\scriptsize $\textit{##3}$} \\[##1] }
  \small
  \begin{center}
  \begin{longtable}{>{$}r<{$} >{$}c<{$} >{$}l<{$} >{$}r<{$}}
}{%
  \end{longtable}
  \end{center}
}

\newcommand{\txss}[1]{\textsuperscript{#1}}

\newcommand{\glist}[1]{\ensuremath{\mathop{\operatorname{\{} #1 \operatorname{\}}}}}

\newcommand{\glistnonempty}[1]{\ensuremath{\mathop{\operatorname{\{} #1 \operatorname{\}}^+}}}

\newcommand{\gopt}[1]{\ensuremath{\mathop{\operatorname{[} #1 \operatorname{]}}}}

\newcommand{\gntn}[1]{\ensuremath{\mathop{\textrm{#1}}}\,}

\newcommand{\gtkw}[1]{\ensuremath{\mathop{\texttt{"#1"}}}\,}
\newcommand{\gtkc}[1]{\ensuremath{\mathop{\texttt{\textquotesingle #1\textquotesingle}}}\,}

\newcommand{\gtregexfull}[2][]{\ensuremath{\mathop{?\, \textrm{\scriptsize /#2/#1} \,?}}}
\newcommand{\gtregex}[2][]{\ensuremath{\textrm{\scriptsize /#2/#1}}}

\newcommand{\gseplist}[2]{#2 \glist{#1 #2}\!\txss{*}}

% Terminals and non-terminals
\newcommand{\gnnl}{\gntn{eol}}
\newcommand{\gneof}{\gntn{eof}}
\newcommand{\gnkeyword}{\gntn{keyword}}
\newcommand{\gnnlkeyword}{\gntn{nl-keyword}}
\newcommand{\gnnlsystemkeyword}{\gntn{nl-sys-keyword}}
\newcommand{\gnnlsubsystemkeyword}{\gntn{nl-subsys-keyword}}
\newcommand{\gnblockend}{\gntn{block-end}}
\newcommand{\gnident}{\gntn{identifier}}
\newcommand{\gnsentence}{\gntn{sentence}}
\newcommand{\gnsentencelist}{\gntn{sentence-list}}
\newcommand{\gnparagraph}{\gntn{paragraph}}
\newcommand{\gnstring}{\gntn{string}}
\newcommand{\gnstringlist}{\gntn{string-list}}

\newcommand{\gncommentchar}{\gtkw{//}}
\newcommand{\gncomment}{\gntn{\ensuremath{\mathcal{C}}}}
\newcommand{\gnlinecomments}{\gntn{\ensuremath{\mathcal{LC}}}}

\newcommand{\gnlando}{\gntn{lando-source}}
\newcommand{\gnspecelem}{\gntn{spec-element}}
\newcommand{\gnclassdict}{\gntn{class-dictionary}}

\newcommand{\gnsystem}{\gntn{system}}
\newcommand{\gnsubsystem}{\gntn{subsystem}}

\newcommand{\gncomponent}{\gntn{component}}
\newcommand{\gncomponentpart}{\gntn{component-part}}
\newcommand{\gnconstraint}{\gntn{constraint}}
\newcommand{\gnquery}{\gntn{query}}
\newcommand{\gncommand}{\gntn{command}}

\newcommand{\gnevent}{\gntn{event}}
\newcommand{\gnscenario}{\gntn{scenario}}
\newcommand{\gnrequirements}{\gntn{requirement}}

\newcommand{\gnname}{\gntn{name}}
\newcommand{\gnnamephrase}{\gntn{name-phrase}}
\newcommand{\gnnameabbrev}{\gntn{name-abbreviation}}
\newcommand{\gnnamephraserel}{\gntn{name-phrase-rel}}
\newcommand{\gnnamelist}{\gntn{name-list}}
\newcommand{\gnexplanation}{\gntn{explanation}}
\newcommand{\gnnamephraselist}{\gntn{name-phrase-list}}
\newcommand{\gnclassdictentry}{\gntn{dictionary-entry}}


\newcommand{\gnevententry}{\gntn{event-entry}}
\newcommand{\gneventdirection}{\gntn{event-direction}}

\newcommand{\gnscenarioentry}{\gntn{scenario-entry}}

\newcommand{\gncreationentry}{\gntn{creation-entry}}

\newcommand{\gnrequirementsentry}{\gntn{req-entry}}

\newcommand{\gnindexing}{\gntn{indexing}}
\newcommand{\gnindexentry}{\gntn{index-entry}}
\newcommand{\gnindexkey}{\gntn{index-key}}
\newcommand{\gnindexvallist}{\gntn{index-val-list}}
\newcommand{\gnindexval}{\gntn{index-val}}

\newcommand{\gnrelkeyword}{\gntn{rel-keyword}}

\newcommand{\gnrelation}{\gntn{relation}}


\title{Lobot Typing Rules}
\author{Matthew Yacavone}

\begin{document}

\maketitle

\section{Grammar}

We use $e$ to denote strings which are enum identifiers, $f$ to denote strings which are field identifiers, $k$ to denote strings which are kind identifiers, and $z$ to denote strings which are integer literals.

\subsection{Expression grammar}

\[ \begin{array}{lclll}
% \exprCtx,\exprCtxAlt
%     &::= & f_1 : A_1 , \ldots , f_n : A_n &&\text{Kind contexts} \\ \\
X   &::= & \BB \| \ZZ \| \{ e_1 , \ldots , e_n \} \| \kw{set}\;\{ e_1 , \ldots , e_n \}
    &&\text{Base types} \\ \\
A,B &::= & X \| \kw{kind}\;\{ F \}
    &&\text{Expression types}  \\ \\
F,G &::= & f_1 : A_1, \ldots, f_n : A_n
    &&\text{Kind fields} \\ \\
t,u &::= & \kw{self} \| f \|
           t.f \| t = u \| t \le u \| t \;\kw{in}\; u \| t \Rightarrow u
    &&\text{Expressions} \\
    &\|  & \neg t \| l \\ \\
l   &::= & \kw{true} \| \kw{false} \| z \| e \| \{ e_1 , \ldots , e_n \}
    &&\text{Literals} \\
    &\|  & \kw{inst}\;\{ f_1 = l_1, \ldots, f_n = l_n \}
\end{array} \]

\subsection{Kind declaration grammar}

On the top level, we permit in-scope kind names in types, thus we must redefine types and kind fields.

\[ \begin{array}{lclll}
\declType, \declTypeAlt
    &::= & X \| \kw{kind}\;\{ \declFields \} \| K
    &&\text{Top-level types} \\ \\
\declFields, \declFieldsAlt
    &::= & f_1 : \declType_1, \ldots, f_n : \declType_n
    &&\text{Top-level kind fields} \\ \\
K   &::= & k \|  k\;K
    &&\text{Kinds and unions of kinds} \\ \\
\hline \\
\declCtx%,\declCtxAlt
    &::= & k_1 : A_1, \ldots, k_n : A_n
    &&\text{Document contexts} \\ \\
D   &::= & \cdot \| k : \declType \text{ where } \{t_1,\ldots,t_n\} \;;\, D
    &&\text{Declarations} \\ \\
\end{array} \]

Note that the `$:$' character in document contexts and declarations means something different than the `$:$' character in kind fields!



\section{Typing Rules}

\subsection{Expression typing rules}

We provide inference rules for the judgement ``$A \vdash t : B$'' meaning ``the expression $t$ has type $B$ in a context of type $A$.'' \\

\fbox{Well-typed expressions}
\begin{mathpar}
\inferrule{ }{A \vdash \kw{self} : A} \and
\inferrule{(f : A) \in F}{\kw{kind}\;\{F\} \vdash f : A} \and
\inferrule{\kw{kind}\;\{F\} \vdash t : \kw{kind}\;\{ G \} \\ \kw{kind}\;\{G\} \vdash f : A }{\kw{kind}\;\{F\} \vdash t.f : A} \\
\inferrule{A \vdash t : B \\ A \vdash u : B}{A \vdash t = u : \BB} \and
\inferrule{A \vdash t : \ZZ \\ A \vdash u : \ZZ}{A \vdash t \le u : \BB} \and
\inferrule{A \vdash t : \{ e_1, \ldots, e_n \} \\ A \vdash u : \kw{set} \; \{ e_1, \ldots, e_n \}}{A \vdash t \;\kw{in}\; u : \BB} \\
\inferrule{A \vdash t : \BB \\ A \vdash u : \BB}{A \vdash t \Rightarrow u : \BB} \and
\inferrule{A \vdash t : \BB}{A \vdash \neg t : \BB} \and
\end{mathpar}

\fbox{Well-typed literals}
\begin{mathpar}
\inferrule{ }{A \vdash \kw{true} : \BB} \and
\inferrule{ }{A \vdash \kw{false} : \BB} \and
\inferrule{ }{A \vdash z : \ZZ} \\
\inferrule{e \in \{ e_1,\ldots,e_n \}}{A \vdash e : \{ e_1,\ldots,e_n \}} \and
%\inferrule{ }{\exprCtx \vdash \{\} : \kw{set}\;\{e_1,\ldots,e_n\}} \and
\inferrule{e_i \in \{ e'_1 , \ldots, e'_m \} \text{ for all } i}{A \vdash \{e_1,\ldots,e_n\} : \kw{set}\;\{ e'_1,\ldots,e'_m \} } \and
\inferrule{ A \vdash l_i : B_i \text{ for all } i }{A \vdash \kw{inst}\;\{ f_1 = l_1, \ldots, f_n = l_n \} : \kw{kind}\;\{ f_1 : B_1, \ldots, f_n : B_n \}}
\end{mathpar}

\subsection{Kind declaration typing rules}

% Typing rules for kind declarations. \\

% \fbox{Well-formed field types}
% \begin{mathpar}
% \inferrule{ }{\declCtx \vdash A \text{ type}} \and
% \inferrule{(k,-) \in \declCtx}{\declCtx \vdash k \text{ type}} \and
% \inferrule{\Gamma_D \vdash U \text{ type} \\ (k,-) \in \declCtx}{\declCtx \vdash U\;k \text{ type}}
% \end{mathpar}

We provide inference rules for two kinds of judgements: 
\begin{itemize}
    \item ``$\declCtx \vdashD D \text{ valid}$'' meaning ``the declarations $D$ are valid in the document context $\declCtx$,''
    %\item ``$\declCtx \vdashD d : A$'' meaning ``the kind declaration $d$ has expression type $A$ in context $\declCtx$''
    \item ``$\declCtx \vdashD \declType \to A$'' meaning ``the top-level type $\declType$ corresponds to the expression type $A$ in the document context $\declCtx$''. This correspondence consists of just resolving all the kind names $k$ present in $\declType$ using the context $\Delta$ (see the final two rules).
\end{itemize}
Given a Lobot document $D$, the goal is to derive $\cdot \vdashD D \text{ valid}$. \\

\fbox{Valid declarations}
\begin{mathpar}
\inferrule{ }{\declCtx \vdashD \cdot \text{ valid}} \and
\inferrule{\declCtx \vdashD \declType \to A \\ A \vdash t_i : \BB \text{ for all } i \\ \declCtx,k : A \vdashD D \text{ valid}}{\declCtx \vdashD k : \declType \text{ where } \{t_1,\ldots,t_n\}\;;\, D \text{ valid}}
\end{mathpar}

% \fbox{Well-typed declarations}
% \begin{mathpar}
% \inferrule{\declCtx \vdashD \declType \to A \\ A \vdash t_i : \BB \text{ for all } i}{\declCtx \vdashD \declType \kw{ where } \{t_1,\ldots,t_n\} : A}
% \end{mathpar}

\fbox{Top-level types to expression types}
\begin{mathpar}
\inferrule{ }{\declCtx \vdashD X \to X} \and
\inferrule{\declCtx \vdashD \declType_i \to A_i \text{ for all } i}{\declCtx \vdashD \kw{kind}\;\{ f_1 : \declType_1, \ldots, f_n : \declType_n \} \to \kw{kind}\;\{ f_1 : A_1, \ldots, f_n : A_n \}} \\
\inferrule{(k : A) \in \declCtx}{\declCtx \vdashD k \to A} \and
\inferrule{(k : A) \in \declCtx \\ \declCtx \vdashD K \to A}{\declCtx \vdashD k\;K \to A} \and
\end{mathpar}

\end{document}
