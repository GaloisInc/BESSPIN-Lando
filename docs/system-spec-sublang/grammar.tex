\documentclass{article}

\usepackage[T1]{fontenc}
\usepackage{longtable}
\usepackage{textcomp}
\usepackage{array}
\usepackage{todonotes}
\usepackage{mathtools}

% Grammar definition helpers
\newenvironment{grammar}{%
  \global\def\grammarVertAdjustment{0mm}
  \newcommand{\gline}{\hline \\[-0.5em]}
  \newcommand{\gcont}{\\ & &}
  \newcommand{\gor}{\ensuremath{\mid}}
  \newcommand{\grule}[4][0mm]{##2 & ::= & ##3 & \mbox{\scriptsize $\textit{##4}$} \\[##1] }
  \newcommand{\gruleplain}[3][0mm]{##2 & \hphantom{::=} & & \mbox{\scriptsize $\textit{##3}$} \\[##1] }
  \small
  \begin{center}
  \begin{longtable}{>{$}r<{$} >{$}c<{$} >{$}l<{$} >{$}r<{$}}
}{%
  \end{longtable}
  \end{center}
}

\newcommand{\txss}[1]{\textsuperscript{#1}}

\newcommand{\glist}[1]{\ensuremath{\mathop{\operatorname{\{} #1 \operatorname{\}}}}}

\newcommand{\glistnonempty}[1]{\ensuremath{\mathop{\operatorname{\{} #1 \operatorname{\}}^+}}}

\newcommand{\gopt}[1]{\ensuremath{\mathop{\operatorname{[} #1 \operatorname{]}}}}

\newcommand{\gntn}[1]{\ensuremath{\mathop{\textrm{#1}}}\,}

\newcommand{\gtkw}[1]{\ensuremath{\mathop{\texttt{"#1"}}}\,}
\newcommand{\gtkc}[1]{\ensuremath{\mathop{\texttt{\textquotesingle #1\textquotesingle}}}\,}

\newcommand{\gtregexfull}[2][]{\ensuremath{\mathop{?\, \textrm{\scriptsize /#2/#1} \,?}}}
\newcommand{\gtregex}[2][]{\ensuremath{\textrm{\scriptsize /#2/#1}}}

\newcommand{\gseplist}[2]{#2 \glist{#1 #2}\!\txss{*}}

% Terminals and non-terminals
\newcommand{\gnnl}{\gntn{eol}}
\newcommand{\gneof}{\gntn{eof}}
\newcommand{\gnkeyword}{\gntn{keyword}}
\newcommand{\gnnlkeyword}{\gntn{nl-keyword}}
\newcommand{\gnnlsystemkeyword}{\gntn{nl-sys-keyword}}
\newcommand{\gnnlsubsystemkeyword}{\gntn{nl-subsys-keyword}}
\newcommand{\gnblockend}{\gntn{block-end}}
\newcommand{\gnident}{\gntn{identifier}}
\newcommand{\gnsentence}{\gntn{sentence}}
\newcommand{\gnsentencelist}{\gntn{sentence-list}}
\newcommand{\gnparagraph}{\gntn{paragraph}}
\newcommand{\gnstring}{\gntn{string}}
\newcommand{\gnstringlist}{\gntn{string-list}}

\newcommand{\gncommentchar}{\gtkw{//}}
\newcommand{\gncomment}{\gntn{\ensuremath{\mathcal{C}}}}
\newcommand{\gnlinecomments}{\gntn{\ensuremath{\mathcal{LC}}}}

\newcommand{\gnlando}{\gntn{lando-source}}
\newcommand{\gnspecelem}{\gntn{spec-element}}
\newcommand{\gnclassdict}{\gntn{class-dictionary}}

\newcommand{\gnsystem}{\gntn{system}}
\newcommand{\gnsubsystem}{\gntn{subsystem}}

\newcommand{\gncomponent}{\gntn{component}}
\newcommand{\gncomponentpart}{\gntn{component-part}}
\newcommand{\gnconstraint}{\gntn{constraint}}
\newcommand{\gnquery}{\gntn{query}}
\newcommand{\gncommand}{\gntn{command}}

\newcommand{\gnevent}{\gntn{event}}
\newcommand{\gnscenario}{\gntn{scenario}}
\newcommand{\gnrequirements}{\gntn{requirement}}

\newcommand{\gnname}{\gntn{name}}
\newcommand{\gnnamephrase}{\gntn{name-phrase}}
\newcommand{\gnnameabbrev}{\gntn{name-abbreviation}}
\newcommand{\gnnamephraserel}{\gntn{name-phrase-rel}}
\newcommand{\gnnamelist}{\gntn{name-list}}
\newcommand{\gnexplanation}{\gntn{explanation}}
\newcommand{\gnnamephraselist}{\gntn{name-phrase-list}}
\newcommand{\gnclassdictentry}{\gntn{dictionary-entry}}


\newcommand{\gnevententry}{\gntn{event-entry}}
\newcommand{\gneventdirection}{\gntn{event-direction}}

\newcommand{\gnscenarioentry}{\gntn{scenario-entry}}

\newcommand{\gncreationentry}{\gntn{creation-entry}}

\newcommand{\gnrequirementsentry}{\gntn{req-entry}}

\newcommand{\gnindexing}{\gntn{indexing}}
\newcommand{\gnindexentry}{\gntn{index-entry}}
\newcommand{\gnindexkey}{\gntn{index-key}}
\newcommand{\gnindexvallist}{\gntn{index-val-list}}
\newcommand{\gnindexval}{\gntn{index-val}}

\newcommand{\gnrelkeyword}{\gntn{rel-keyword}}

\newcommand{\gnrelation}{\gntn{relation}}


\usepackage{cmbright}
%\usepackage[varg]{txfonts}
%\usepackage{pxfonts}

\begin{document}

\section{Notation}

The grammar for the Lando System Specification Sublanguage is written in the EBNF notation. The main elements of the notation that we utilize are:
\begin{itemize}
  \item Terminals are represented with double or single quotes; e.g. $\gtkw{explanation}$.
  \item Optional bits are represented with squared brackets; e.g. $\gopt{\gtkw{explanation} \gnparagraph}$.
  \item Repetition is represented with curly braces; e.g. $\glist{\gnident}$
  \item We use a slightly enhanced notation - $\glistnonempty{a}$ to indicate \emph{non-zero} repetitions. This is simply equivalent to: $a \glist{a}$.
  \item For defining terminals, we would like to use the EBNF special form to declare an extended regular expression - for example: $\gtregexfull{\textbackslash w\\+}$. However since this is rather verbose, we will simply use $\gtregex{\textrm{\textbackslash w\\+}}$ for convenience.
  \item Blocks in the language are typically delimited by keywords indicating the start of \emph{another} block. In defining the grammar this translates to lookaheads: (i.e. peeking at incoming tokens without consuming them). We use the perl regular expression format to indicate this. E.g. $\gtregex{?= (\gnnl \gtkw{system})}$.
\end{itemize}

\section{Grammar}

\begin{grammar}
  \grule{\gnlando}{
    \glist{\gnspecelem}
  }{Lando source}
  
  \grule{\gnspecelem}{
    \gnsystem \gor \gnsubsystem \gor \gncomponent \gor \gnevent \gcont 
    \gor \gnscenario \gor \gnrequirements
  }{Specification Elements}

  %\grule{\gnclassdict}{
    %\gtkw{dictionary} \gnname 
    %\glistnonempty{\gnclassdictentry} 
    %\gtkw{end} %TODO
  %}{Class Dictionary}

  %\grule[0.5em]{\gnclassdictentry}{
    %\gtkw{class} \gnname 
    %\gtkw{description} \gnparagraph %TODO: What is the last string?
  %}{Class Dictionary Entry}

  \gline

  \grule{\gnsystem}{
    \gtkw{system} \gnnamephraserel \gopt{\gnrelkeyword \gnnamephrase} \gnnl \gcont
    \; \gnexplanation \gnnl \gcont
    \; \gopt{\gtkw{indexing} \gnnl \gnindexing \gnnl} \gcont
    \; \gseplist{\gnnl}{\gnsubsystem} \gcont
    \; \gtregex{(?= \gnnlsystemkeyword | \gneof)} \gnblockend
  }{System}

  \gline

  \grule{\gnsubsystem}{
    \gtkw{subsystem} \gnnamephraserel \gopt{\gnrelkeyword \gnnamephrase} \gnnl \gcont
    \; \gnexplanation \gnnl \gcont
    \; \gopt{\gtkw{indexing} \gnnl \gnindexing \gnnl} \gcont
    \; \gseplist{\gnnl}{\gncomponent} \gcont
    \; \gtregex{(?= \gnnlsubsystemkeyword | \gneof)} \gnblockend
  }{Cluster}

  \gline

  \grule{\gncomponent}{
    \gtkw{component} \gnnamephraserel \gopt{\gnrelkeyword \gnnamephrase} \gnnl \gcont
    \; \gnexplanation \gnnl \gcont
    \; \gseplist{\gnnl}{\gncomponentpart} \gcont
    \; \gtregex{(?= \gnnlkeyword | \gneof)} \gnblockend
  }{Class}

  \grule{\gncomponentpart}{
    \gnconstraint | \gncommand | \gnquery
  }{Component Parts}

  \grule{\gnconstraint}{
    \gtregex[m]{[\text{\textasciicircum}.?!]\txss{+}?\,\textbackslash.]}
  }{Constraint}

  \grule{\gnquery}{
    \gtregex[m]{[\text{\textasciicircum}.?!]\txss{+}?\,\textbackslash?]}
  }{Query}

  \grule[0.5em]{\gncommand}{
    \gtregex[m]{[\text{\textasciicircum}.?!]\txss{+}?\,!]}
  }{Command}

  \gline

  \grule{\gnevent}{
    \gtkw{events} \gnnamephrase \gnnl \gcont
    \; \gseplist{\gnnl}{\gnevententry} \gcont
    \; \gtregex{(?= \gnnlkeyword | \gneof)} \gnblockend
  }{Events}

  \grule{\gnevententry}{
    \gnident \gnnl \gnsentence
  }{Event Entry}

  \gline

  \grule{\gnscenario}{
    \gtkw{scenarios} \gnnamephrase \gnnl \gcont
    \; \gseplist{\gnnl}{\gnevententry} \gcont
    \; \gtregex{(?= \gnnlkeyword | \gneof)} \gnblockend
  }{Scenario}

  \grule{\gnscenarioentry}{
    \gnident \gnnl \gnsentence
  }{Scenario Entry}

  \gline


  \grule{\gnrequirements}{
    \gtkw{requirements} \gnnamephrase \gnnl \gcont
    \; \gseplist{\gnnl}{\gnrequirementsentry} \gcont
    \; \gtregex{(?= \gnnlkeyword | \gneof)} \gnblockend
  }{Requirements}

  \grule{\gnrequirementsentry}{
    \gnident \gnnl \gnsentence
  }{Requirements Entry}

  \gline
  
  \grule{\gnindexing}{
    \gseplist{\gnnl}{\gnindexentry}
  }{Index List}

  
  \grule{\gnindexentry}{
    \gnindexkey \gtkc{:} \gnindexvallist
  }{Index List}

  \grule{\gnindexkey}{
    \gtregex{[\text{\textasciicircum:}]\txss{+}}
  }{Index Key}

  \grule{\gnindexvallist}{
    \gseplist{\gnnl}{\gnindexval} \gcont
    \; \gtregex{(?= \gneof | \gnnlkeyword | \gnnl \gnindexkey )}
  }{Index Value List}

  \grule[0.5em]{\gnindexval}{
    \gtregex{[\text{\textasciicircum:}]\txss{+}}
  }{Index Value}
  

  \gline

  \grule{\gnname}{\gnident}{Name}
  \grule{\gnnamephraserel}{
    \gtregex{\textbackslash{}w[\textbackslash{}w\textbackslash{}s]*\,(?= \gnrelkeyword | \gnnl)}
  }{Name-Phrase}
  \grule{\gnnamelist}{\gnname \glist{\gtkc{,} \gnname}}{Name List}
  \grule{\gnstring}{\gtregex{[\textasciicircum,]\txss{+}?}}{String}
  \grule{\gnstringlist}{\gnstring \glist{, \gnstring}}{List of Strings}
  \grule{\gnsentence}{\gtregex[m]{[\text{\textasciicircum}.?!]\txss{+}? [.?!]}}{Sentence}
  \grule{\gnsentencelist}{\gnsentence \glist{\gtkc{,} \gnsentence}}{String List}
  \grule{\gnparagraph}{\gnsentence^+ \gtregex{(?= (\gnnl \gnkeyword | \gneof))}}{Paragraph}
  \grule{\gnexplanation}{\gnparagraph}{Explanation}
  \grule{\gnkeyword}{<all keywords>}{All\ Keywords}
  \grule{\gnnlkeyword}{\gnnl \, \gnkeyword}{Keyword on new line}
  \grule{\gnnlsystemkeyword}{\gnnl \, \gtkw{system}}{}
  \grule{\gnnlsubsystemkeyword}{\gnnl \, (\gtkw{subsystem} | \gtkw{system})}{}
  \grule{\gnrelkeyword}{\gtkw{inherit} | \gtkw{client} | \gtkw{contains}}{Relation keywords}
  \grule{\gnblockend}{\gnnl | \gneof}{Block End}
  \grule{\gnident}{\gtregex{\textrm{\textbackslash w+}}}{Identifier}
  \grule{\gnnl}{}{New Line}
\end{grammar}


\end{document}

