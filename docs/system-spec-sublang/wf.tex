\documentclass{article}
\usepackage{fullpage}
\usepackage[T1]{fontenc}
\usepackage[utf8]{inputenc}
\usepackage{longtable}
\usepackage{textcomp}
\usepackage{array}
\usepackage{todonotes}
\usepackage{mathtools}
\usepackage{xparse}
\usepackage{mathpartir}
\usepackage{amssymb}

%\input{macros2}

\begin{document}

\title{Lando SSI Well-formedness}
\maketitle

\section{Raw AST Grammar}

This is intended to correspond precisely to the {\tt RawAST} structure built by the current parser.

\newcommand{\nt}[1]{\ensuremath{\mathit{#1}}}
\newcommand{\lab}[1]{\ensuremath{\texttt{#1}}}
\newcommand{\listof}[1]{\ensuremath{[{#1}]}}
\newcommand{\nillist}[1]{[]_{#1}}
\newcommand{\conslist}[2]{{#1}::{#2}}
\newcommand{\optof}[1]{\ensuremath{{#1}?}}
\newcommand{\some}[1]{\lceil{#1}\rceil}
\newcommand{\none}[1]{\cdot_{#1}}
\newcommand{\proj}[2]{\ensuremath{{#1}.{#2}}}
\newcommand{\emptym}{\emptyset} 

\[
\begin{array}{rcl}
  s \in \nt{source} & \rightarrow &\lab{Source}\{\lab{uid}:\nt{uid},\lab{elems}:\listof{\nt{elem}}, \lab{relations}:\listof{\nt{relation}}\} \\
  e \in \nt{elem} & \rightarrow & \lab{System}\{\lab{uid}:\nt{uid},\lab{name}:\nt{name},\lab{abbrev}:\optof{\nt{name}},
                                                \lab{expanation}:\nt{text},\lab{indexings}:\listof{\nt{indexing}}\} \\
                  & \rightarrow & \lab{Subsystem}\{\lab{uid}:\nt{uid},\lab{name}:\nt{name},\lab{abbrev}:\optof{\nt{name}},
                                                \lab{expanation}:\nt{text},\lab{indexings}:\listof{\nt{indexing}}\} \\
                  & \rightarrow &\lab{Component}\{\lab{uid}:\nt{uid},\lab{name}:\nt{name},\lab{abbrev}:\optof{\nt{name}},
                                                \lab{expanation}:\nt{text},\lab{parts}:\listof{\nt{part}}\} \\
                  & \rightarrow & \lab{Events}\{\lab{uid}:\nt{uid},\lab{name}:\nt{name},\lab{contents}:\listof{\nt{item}}\} \\
                  & \rightarrow & \lab{Scenarios}\{\lab{uid}:\nt{uid},\lab{name}:\nt{name},\lab{contents}:\listof{\nt{item}}\} \\
                  & \rightarrow & \lab{Requirements}\{\lab{uid}:\nt{uid},\lab{name}:\nt{name},\lab{contents}:\listof{\nt{item}}\} \\
  x \in \nt{indexing} & \rightarrow & \lab{Indexing}\{\lab{key}:\nt{text},\lab{values}:\listof{\nt{text}}\} \\
  p \in \nt{part} & \rightarrow & \lab{Constraint}\{\lab{contents}:\nt{text}\} \\
                  & \rightarrow & \lab{Query}\{\lab{contents}:\nt{text}\} \\
                  & \rightarrow & \lab{Command}\{\lab{contents}:\nt{text}\} \\
  i \in \nt{item} & \rightarrow & \lab{Item}\{\lab{id}:\nt{name},\lab{value}:\nt{text}\} \\
  r \in \nt{relation} & \rightarrow & \lab{Inherit}\{\lab{sub}:\nt{name},\lab{super}:\nt{name}\} \\
                      & \rightarrow & \lab{Contains}\{\lab{child}:\nt{name},\lab{parent}:\nt{name}\} \\
                      & \rightarrow & \lab{ImplicitContains}\{\lab{child}:\nt{uid},\lab{parent}:\nt{uid}\} \\
                      & \rightarrow & \lab{Client}\{\lab{client}:\nt{name},\lab{provider}:\nt{name}\} \\
  u \in \nt{uid} & & \text{(base type)} \\
  t \in \nt{text}& & \text{(base type)} \\
  n \in \nt{name}& & \text{(base type)} \\
\end{array}
\]
  

Meta-syntax conventions:
Lists of $t$ are written $\listof{t}$; $\nillist{t}$ is an empty list of $t$'s and $\conslist{x}{xs}$ is the list with head $x$ and tail $xs$.
An optional $t$ is written $\optof{t}$; a present option value $x$ is written $\some{x}$ and a missing option value of type $t$ is written $\none{t}$.
We write $\proj{x}{y}$ for the projection of field $y$ from a node $x$ .
In the pattern $x@\lab{Foo}\{\ldots\}$, $x$ is bound to the entire node labeled $\lab{Foo}$.
The pattern $\lab{(Foo|Bar)}\{\ldots\}$ matches nodes labeled either $\lab{Foo}$ or $\lab{Bar}$.
Disjoint union of maps is written $\uplus$, and is undefined if the domains of the maps are not disjoint.  An empty map is written $\emptym$.
A judgement applies only if all premises are fully defined.

\vfill\eject

\section{Well-formedness Judgements}

\subsection{Uniqueness of UIDs}

\newcommand{\uidmap}{\Phi}

These judgements check that each \nt{uid} is indeed unique and
produces a finite map $\uidmap : \nt{uid} \rightarrow \nt{elem} + \nt{source}$
from each defined \nt{uid} to its associated element definition (or
the top level $\nt{source}$).

\newcommand{\blduidmap}[2]{\vdash {#1} \Rightarrow {#2}}

\begin{mathparpagebreakable}

\inferrule{
  \blduidmap{es}{\uidmap}  \\
  \uidmap' = \uidmap \uplus \{ u \mapsto s \} \\
}{
  \blduidmap{s@\lab{Source}\{\lab{uid}=u,\lab{elems}=es,\ldots\}}{\uidmap'}
}
            
\\

\inferrule{
}{
  \blduidmap{\nillist{\nt{elem}}}{\emptym}
}

\inferrule{
  \blduidmap{es}{\uidmap} \\
  \uidmap' = \uidmap \uplus \{ \proj{e}{\lab{uid}} \mapsto e  \} \\
}
{  \blduidmap{\conslist{e}{es}}{\uidmap'}
  }

\end{mathparpagebreakable}


\subsection{Implicit Container Relations}

\newcommand{\icmap}{\Sigma}

\newcommand{\bldicmap}[3]{{#1} \vdash {#2} \Rightarrow {#3}}

\newcommand{\reachable}[2]{\texttt{reachable}_{#1}({#2})}
% Reachability should be reflexive.
\newcommand{\nocycles}[1]{\texttt{nocycles}({#1})}
\newcommand{\validic}[2]{\texttt{valid-impl-contains}({#1},{#2})}

These judgments check that the implicit container relations clauses are well formed
and produce a map $\icmap: \nt{uid} \rightarrow \nt{uid}$ mapping each
implicitly contained element to its immediately containing element.

\begin{mathparpagebreakable}
  \inferrule{
  }{\bldicmap{\uidmap}{\nillist{\nt{relation}}}{\emptym}
  }
    
\inferrule{
  \bldicmap{\uidmap}{rs}{\icmap} \\
%  u_c \neq u_p \\  not needed if reachability is reflexive
  \validic{\uidmap(u_c)}{\uidmap(u_p)} \\
  u_c \not\in \reachable{\icmap}{u_p} \\
  \icmap' = \icmap \uplus \{ u_c \mapsto u_p \} \\
}{
  \bldicmap{\uidmap}{\conslist{\lab{ImplicitContains}\{\lab{child}=u_c,\lab{parent}=u_p\}}{rs}}{\icmap'}
}

\inferrule{
  \bldicmap{\uidmap}{rs}{\icmap} \\
}{
  \bldicmap{\uidmap}{\conslist{\lab{Contains}\{\ldots\}}{rs}}{\icmap}
}

\inferrule{
  \bldicmap{\uidmap}{rs}{\icmap} \\
}{
  \bldicmap{\uidmap}{\conslist{\lab{Inherit}\{\ldots\}}{rs}}{\icmap}
}

\inferrule{
  \bldicmap{\uidmap}{rs}{\icmap} \\
}{
  \bldicmap{\uidmap}{\conslist{\lab{Client}\{\ldots\}}{rs}}{\icmap}
}


\inferrule{
}{\validic{\lab{Subsystem}\{\ldots\}}{\lab{System}\{\ldots\}}
}

\inferrule{
}{\validic{\lab{(Subsystem|Component|Scenarios|Requirements|Events)}\{\ldots\}}{\lab{Subsystem}\{\ldots\}}
}

\\

\inferrule{
}{x \in \reachable{M}{x}}

\inferrule{
  M(z) = x \\
  z \in \reachable{M}{y}
}{x \in \reachable{M}{y}
}


\end{mathparpagebreakable}

\subsection{Binding Contexts}

\newcommand{\con}{\Gamma}
\newcommand{\iccon}{\Delta}

These judgements check that the set of names 
defined by each implicit container has no duplicates,
and so uniquely defines a context $\con: \nt{name} \rightarrow \nt{uid}$.
The judgements 
produce a map $\iccon: \nt{uid} \rightarrow \con$
from each implicit container to the context it defines.

\newcommand{\bldcon}[2]{\vdash {#1} \Rightarrow {#2}}
\newcommand{\bldiccon}[3]{{#1} \vdash {#2} \Rightarrow {#3}}

\begin{mathparpagebreakable}

\inferrule{
  n_1 \neq n_2
}{
  \bldcon{\lab{(System|Subsystem|Component)}\{\lab{uid}=u,\lab{name}=n_1,\lab{abbrev}=\some{n_2},\ldots\}}{\{ n_1 \mapsto u, n_2 \mapsto u \}}
}

\inferrule{
}{
  \bldcon{\lab{(System|Subsystem|Component)}\{\lab{uid}=u,\lab{name}=n,\lab{abbrev}=\none{\nt{name}},\ldots\}}{\{ n \mapsto u\}}
}

\inferrule{
}{
  \bldcon{\lab{(Events|Scenarios|Requirements)}\{\lab{uid}=u,\lab{name}=n,\ldots\}}{\{ n \mapsto u\}}
}


\inferrule{
    \bldiccon{\icmap}{es}{\iccon} \\
    u_p = \icmap(\proj{e}{\lab{uid}}) \\
    \con_0 = \left\{{\begin{array}{ll}
                       \iccon(u_p)&\text{if}~u_p \in Dom(\iccon)\\
                       \emptym & \text{otherwise}
                   \end{array}}\right. \\\\
    \vdash e : \con \\
    \iccon' = \iccon[u_p \mapsto (\con_0 \uplus \con)] \\
}{\bldiccon{\icmap}{\conslist{e}{es}}{\iccon'}
}

\inferrule{
  \bldiccon{\icmap}{es}{\iccon}
}{
  \bldiccon{\icmap}{\lab{Source}\{\lab{elems}=es,\ldots\}}{\iccon}
}
  

\end{mathparpagebreakable}

\subsection{Explicit Relations}

\newcommand{\inhmap}{I}
\newcommand{\xcmap}{K}
\newcommand{\clmap}{\Pi}
\newcommand{\bldermap}[6]{{#1},{#2} \vdash {#3} \Rightarrow ({#4},{#5},{#6})}

\newcommand{\validinh}[2]{\texttt{valid-inherit}({#1},{#2})}
\newcommand{\validcontains}[2]{\texttt{valid-contains}({#1},{#2})}
\newcommand{\validclient}[2]{\texttt{valid-client}({#1},{#2})}

These judgements check that the explicit relations are well-formed,
and build an inheritance map
$\inhmap: \nt{uid} \rightarrow \nt{uid}$ from sub-things to super-things,
an explicit containment map
$\xcmap: \nt{uid} \rightarrow \nt{uid}$ from children to parents,
and a client relation
$\clmap: \nt{uid} \times \nt{uid}$ between clients and providers.

\emph{
  TBD:
  \begin{itemize}
  \item Unclear where we will get the contexts $\con$.
  \item Need to fill in the validity definitions.
  \item Treating $\inhmap$ as a map implies single inheritance.
  \item Treating $\xcmap$ as a map implies a thing is contained in at most one other thing.
  \item Need to consider interction of implicit and explicit relation maps, in particular to find cycles across the combination of the two.
  \item Calculation of $\clmap$ doesn't check for duplicate declarations of a given relation; should it?
  \end{itemize}
}


\begin{mathparpagebreakable}

\inferrule{
}{
  \bldermap{\uidmap}{\con}{\nillist{\nt{relation}}}{\emptym}{\emptym}{\emptym}
}

\inferrule{
  \bldermap{\uidmap}{\con}{rs}{\inhmap}{\xcmap}{\clmap} \\
  u_{sub} = \con(n_{sub}) \\
  u_{super} = \con(n_{super}) \\
  u_{sub} \not \in \reachable{\inhmap}{u_{super}} \\
  \validinh{\uidmap(u_{sub})}{\uidmap(u_{super})} \\
  \inhmap' = \inhmap \uplus \{ u_{sub} \mapsto u_{super} \}
}{
  \bldermap{\uidmap}{\con}{\conslist{\lab{Inherit}\{\lab{sub}=n_{sub},\lab{super}=n_{super}\}}{rs}}{\inhmap'}{\xcmap}{\clmap}
}


\inferrule{
  \bldermap{\uidmap}{\con}{rs}{\inhmap}{\xcmap}{\clmap} \\
  u_c = \con(n_c) \\
  u_p = \con(n_p) \\
  u_c \not \in \reachable{\xcmap}{u_p} \\
  \validcontains{\uidmap(u_c)}{\uidmap(u_p)} \\
  \xcmap' = \xcmap \uplus \{ u_c \mapsto u_p \}
}{
  \bldermap{\uidmap}{\con}{\conslist{\lab{Contains}\{\lab{child}=n_c,\lab{parent}=n_p\}}{rs}}{\inhmap}{\xcmap'}{\clmap}
}

          \\
          
\inferrule{
  \bldermap{\uidmap}{\con}{rs}{\inhmap}{\xcmap}{\clmap} \\
  u_c = \con(n_c) \\
  u_p = \con(n_p) \\
  \validclient{\uidmap(u_c)}{\uidmap(u_p)} \\
  \clmap' = \clmap \cup \{ (u_c,u_p) \}
}{
  \bldermap{\uidmap}{\con}{\conslist{\lab{Client}\{\lab{client}=n_c,\lab{provider}=n_p\}}{rs}}{\inhmap}{\xcmap}{\clmap'}
}


\\

\inferrule{
  \bldermap{\uidmap}{\con}{rs}{\inhmap}{\xcmap}{\clmap} \\
}{
  \bldermap{\uidmap}{\con}{\conslist{\lab{ImplicitContains}\{\ldots\}}{rs}}{\inhmap}{\xcmap}{\clmap}
}
          
\end{mathparpagebreakable}

\subsection{All Together}

\emph{
  TBD:
  \begin{itemize}
  \item Use of $\con = \iccon(u)$ is bogus; need to fix once we understand desired scoping rules better.
  \item Need to describe interaction of $\icmap$ and $\xcmap$ and perhaps build their union.
  \end{itemize}
}
            
\begin{mathparpagebreakable}

\inferrule{
  \blduidmap{s}{\uidmap} \\
  \bldicmap{\uidmap}{rs}{\icmap} \\
  \bldiccon{\icmap}{s}{\iccon} \\
  \con = \iccon(u) \\
  \bldermap{\uidmap}{\con}{rs}{\inhmap}{\xcmap}{\clmap} \\\\
  \text{$\exists$ at most one $e \in es$ s.t. $e = \lab{System}\{\ldots\}$} \\
}{
  \vdash s@\lab{Source}\{\lab{uid}=u,\lab{elems}=es,\lab{relations}=rs\} \Rightarrow (\uidmap,\icmap,\iccon,\inhmap,\xcmap,\clmap)
}
\end{mathparpagebreakable}


\end{document}


