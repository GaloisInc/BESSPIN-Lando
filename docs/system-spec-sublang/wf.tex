\documentclass{article}
\usepackage{fullpage}
\usepackage[T1]{fontenc}
\usepackage[utf8]{inputenc}
\usepackage{longtable}
\usepackage{textcomp}
\usepackage{array}
\usepackage{todonotes}
\usepackage{mathtools}
\usepackage{xparse}
\usepackage{mathpartir}
\usepackage{amsmath}
\usepackage{amssymb}

%\input{macros2}

\begin{document}

\title{Lando SSL Version 2 Well-formedness}
\maketitle

\section{Notation}

\newcommand{\nt}[1]{\ensuremath{\mathit{#1}}}
\newcommand{\lab}[1]{\ensuremath{\texttt{#1}}}
\newcommand{\listof}[1]{\ensuremath{[{#1}]}}
\newcommand{\nillist}[1]{[]_{#1}}
\newcommand{\conslist}[2]{{#1}::{#2}}
\newcommand{\singlelist}[1]{[{#1}]}
\newcommand{\listlast}[1]{\mathit{last}({#1})}
\newcommand{\optof}[1]{\ensuremath{{#1}?}}
\newcommand{\some}[1]{\lceil{#1}\rceil}
\newcommand{\none}[1]{\cdot_{#1}}
\newcommand{\proj}[2]{\ensuremath{{#1}.{#2}}}
\newcommand{\emptym}{\emptyset} 

Lists of $t$ are written $\listof{t}$; $\nillist{t}$ is an empty list of $t$'s, $\conslist{x}{xs}$ is the list with head $x$ and tail $xs$,
and $\singlelist{x}$ is the singleton list containing $x$.
$\listlast{l}$ returns the last element of (necessarily non-empty) list $l$.
An optional $t$ is written $\optof{t}$; a present option value $x$ is written $\some{x}$ and a missing option value of type $t$ is written $\none{t}$.
We write $\proj{x}{y}$ for the projection of field $y$ from a node $x$ .
In the pattern $x@\lab{Foo}\{\ldots\}$, $x$ is bound to the entire node labeled $\lab{Foo}$.
The pattern $\lab{(Foo|Bar)}\{\ldots\}$ matches nodes labeled either $\lab{Foo}$ or $\lab{Bar}$.
Disjoint union of maps is written $\uplus$, and is undefined if the domains of the maps are not disjoint.  An empty map is written $\emptym$.
A judgement applies only if all premises are fully defined.
\section{Raw AST Grammar}

This is intended to correspond precisely to the {\tt RawAST} structure built by the current parser, omitting source position tags
and comments. The AST corresponds directly to the concrete syntax, with one exception: implicit bodies of {\tt system}s and {\tt subsystem}s are
converted into explicit ones. 

\[
\begin{array}{rcll}
  s \in \nt{source} & \rightarrow &\lab{Source}\{\lab{elems}:\listof{\nt{elem}}\} \\
  e \in \nt{elem} & \rightarrow & \lab{System}\{\lab{name}:\nt{name},\lab{abbrev}:\optof{\nt{name}},
                                                \lab{explanation}:\nt{text},\lab{indexings}:\listof{\nt{indexing}},
                                                \lab{body}:\listof{\nt{elem}}\} \\
                  & \rightarrow & \lab{Subsystem}\{\lab{name}:\nt{name},\lab{abbrev}:\optof{\nt{name}},
                                                \lab{inherits}:\listof{\nt{qname}},\lab{clientOf}:\listof{\nt{qname}}, \\
                  &             & \ \ \ \ \ \ \ \ \ \ \ \ \ \ \ \                                                 
                                                \lab{explanation}:\nt{text},\lab{indexings}:\listof{\nt{indexing}},
                                                \lab{body}:\listof{\nt{elem}}\} \\
                  & \rightarrow & \lab{SubystemImport}\{\lab{name}:\nt{qname},\lab{abbrev}:\optof{\nt{name}},
                                                        \lab{clientOf}:\listof{\nt{qname}}\} \\
                  & \rightarrow &\lab{Component}\{\lab{name}:\nt{name},\lab{abbrev}:\optof{\nt{name}},
                                                \lab{inherits}:\listof{\nt{qname}},\lab{clientOf}:\listof{\nt{qname}},\\
                  &             & \ \ \ \ \ \ \ \ \ \ \ \ \ \ \ \                                                 
                                                \lab{explanation}:\nt{text},\lab{parts}:\listof{\nt{part}}\} \\
                  & \rightarrow & \lab{ComponentImport}\{\lab{name}:\nt{qname},\lab{abbrev}:\optof{\nt{name}},
                                                         \lab{clientOf}:\listof{\nt{qname}}\} \\
                  & \rightarrow & \lab{Events}\{\lab{name}:\nt{name},\lab{contents}:\listof{\nt{item}}\} \\
                  & \rightarrow & \lab{Scenarios}\{\lab{name}:\nt{name},\lab{contents}:\listof{\nt{item}}\} \\
                  & \rightarrow & \lab{Requirements}\{\lab{name}:\nt{name},\lab{contents}:\listof{\nt{item}}\} \\
                  & \rightarrow & \lab{Relation}\{\lab{name}:\nt{qname},\lab{inherits}:\listof{\nt{qname}},
                                                  \lab{clientOf}:\listof{\nt{qname}}\} \\
  x \in \nt{indexing} & \rightarrow & \lab{Indexing}\{\lab{key}:\nt{text},\lab{values}:\listof{\nt{text}}\} \\
  p \in \nt{part} & \rightarrow & \lab{Constraint}\{\lab{contents}:\nt{text} \} \\
  & \rightarrow & \lab{Query}\{\lab{contents}:\nt{text}\} \\
                  & \rightarrow & \lab{Command}\{\lab{contents}:\nt{text}\} \\
  i \in \nt{item} & \rightarrow & \lab{Item}\{\lab{id}:\nt{name},\lab{value}:\nt{text} \} \\
  u \in \nt{uid} & & \text{(base type)} \\
  t \in \nt{text}& & \text{(base type)} \\
  n \in \nt{name}& & \text{(base type)} \\
  q \in \nt{qname}& = & \listof{\nt{name}} \\
  c \in \nt{comment}&  & \text{(base type)} \\
\end{array}
\]
  
\vfill\eject

\section{Well-formedness Judgements}

\newcommand{\senv}{\Phi}
\newcommand{\env}{\Gamma}
\newcommand{\qlookup}[2]{\mathit{qlook}_{#1}({#2})}
\newcommand{\qnames}[1]{\mathit{qnames}({#1})}
\newcommand{\nocycles}[1]{\mathit{nocycles}({#1})}

Although the AST does not explicitly include element labels, we assume that each node (in particular each \nt{elem}) 
has a unique identity with a well-defined notion of equality.

$\env : \nt{name} \rightarrow \nt{elem}$ is an environment (finite map) from names
to their corresponding elements.

$\senv : \nt{elem} \rightarrow \env$ is a finite map from (sub)systems
to the environments they generate. 

The function $\qlookup{\env}{q}$ resolved a qualified name $q$ starting in environment $\env$; it is defined thus:
\[
\begin{array}{rcl}
\qlookup{\env}{\singlelist{n}} &= &\env(n) \\
\qlookup{\env}{\conslist{n}{ns}} & =& \qlookup{\senv(\env(n))}{ns}
\end{array}
\]

$I : \nt{elem} \times \nt{elem}$ is an inheritance relation between elements, where $(e_1,e_2) \in I$ means
$e_1$ inherits from $e_2$. The predicate $\nocycles{I}$ holds when $I$ has no cycles. 

We assume the existence of a function $\qnames{t}$ that returns the list of \nt{qname}s mentioned in \nt{text} $t$.


\newcommand{\validinh}[2]{\texttt{valid-inherit}({#1},{#2})}
\newcommand{\validclient}[2]{\texttt{valid-client}({#1},{#2})}
\newcommand{\validcontains}[2]{\texttt{valid-contains}({#1},{#2})}

\newcommand{\wb}[3]{{#1} \vdash {#2} \Rightarrow {#3}}
\newcommand{\wbx}[2]{{#1} \vdash {#2}}

A specification $s$ is well-formed if it is possible to find a top-level environment $\env_0$, a global $\senv_0$,
and a global inheritance relation $I_0$ (all implicitly threaded everywhere) such that $\wbx{}{s}$.

\begin{mathparpagebreakable}


\inferrule{
  \wb{\env_0}{es}{\env_0} \\
  \nocycles{I} \\
  \forall e_1,e_2 \in es, e_1 = \lab{System}\{\ldots\} \land e_2 = \lab{System}\{\ldots\} \implies e_1 = e_2
}{
  \wbx{}{\lab{Source}\{\lab{elems}=es\}}
}
            
\inferrule{
  \wb{\env}{e}{\env'} \\
  \wb{\env}{es}{\env''} 
}{
  \wb{\env}{\conslist{e}{es}}{\env' \uplus \env''}
}

\inferrule{
}{
  \wb{\env}{\nillist{\nt{elem}}}{\{\}}
}


\inferrule{
  \wbx{\env}{t} \\
  % \forall x \in xs, \wb{\env}{x} \\
  \forall e \in es_b, \validcontains{e_s}{e} \\
  \senv_0(e_s) = \env' \\
  \wb{\env \uplus \env'}{es_b}{\env'} \\
}{
  \wb{\env}{e_s@\lab{System}\{\lab{name}=n,\lab{abbrev}=\some{n_a},\lab{explanation}=t,\lab{body}=es_b,\ldots\}}{\{n \mapsto e_s,n_a \mapsto e_s\}}
}

\inferrule{
  \wbx{\env}{t} \\
  % \forall x \in xs, \wb{\env}{x} \\
  \forall e \in es_b, \validcontains{e_s}{e} \\
  \senv_0(e_s) = \env' \\
  \wb{\env \uplus \env'}{es_b}{\env'} \\
}{
  \wb{\env}{e_s@\lab{System}\{\lab{name}=n,\lab{abbrev}=\none{\nt{name}},\lab{explanation}=t,\lab{body}=es_b,\ldots\}}{\{n \mapsto e_s\}}
}

\inferrule{
  \forall q \in qs_i, \validinh{e_s}{\qlookup{\env}{q}} \\
  \forall q \in qs_c, \validclient{e_s}{\qlookup{\env}{q}} \\
  \wbx{\env}{t} \\
  \forall e \in es_b, \validcontains{e_s}{e} \\
  % \forall x \in xs, \wb{\env}{x} \\
  \senv_0(e_s) = \env' \\
  \wb{\env \uplus \env'}{es_b}{\env'} \\
}
{
  \wb{\env}{e_s@\lab{Subsystem}\left\{\begin{array}{ll}
      \lab{name}=n,\lab{abbrev}=\some{n_a},\lab{inherits}=qs_i, \\
      \lab{clientOf}= qs_c,\lab{explanation}=t,\lab{body}=es_b,\ldots
      \end{array}\right\}}
     {\{n \mapsto e_s,n_a \mapsto e_s\}}
}

\inferrule{
  \forall q \in qs_i, \validinh{e_s}{\qlookup{\env}{q}} \\
  \forall q \in qs_c, \validclient{e_s}{\qlookup{\env}{q}} \\
  \wbx{\env}{t} \\
  \forall e \in es_b, \validcontains{e_s}{e} \\
  % \forall x \in xs, \wb{\env}{x} \\
  \senv_0(e_s) = \env' \\
  \wb{\env \uplus \env'}{es_b}{\env'} \\
}
{
  \wb{\env}{e_s@\lab{Subsystem}\left\{\begin{array}{ll}
    \lab{name}=n,\lab{abbrev}=\none{\nt{name}}, \lab{inherits}=qs_i, \\
    \lab{clientOf}= qs_c,\lab{explanation}=t,\lab{body}=es_b,\ldots
    \end{array}\right\}}
    {\{n \mapsto e_s\}}
}
            
\inferrule{
  \qlookup{\env_0}{q} = e \\
  \forall q \in qs_c, \validclient{e}{\qlookup{\env}{q}} \\
}
{
  \wb{\env}{e_s@\lab{SubsystemImport}\{\lab{name}=q,\lab{abbrev}=\some{n_a},\lab{clientOf}= qs_c\}}{\{n_a \mapsto e_s\}}
}

\inferrule{
  \qlookup{\env_0}{q} = e \\
  \forall q \in qs_c, \validclient{e}{\qlookup{\env}{q}} \\
}
{
  \wb{\env}{e_s@\lab{SubsystemImport}\{\lab{name}=q,\lab{abbrev}=\none{\nt{name}},\lab{clientOf}= qs_c\}}{\{\listlast{q}\mapsto e_s\}}
}


\inferrule{
  \forall q \in qs_i, \validinh{e_c}{\qlookup{\env}{q}} \\
  \forall q \in qs_c, \validclient{e_c}{\qlookup{\env}{q}} \\
  \wbx{\env}{t} \\
  \forall p \in ps, \wbx{\env}{p}  \\
}
{
  \wb{\env}{e_c@\lab{Component}\left\{\begin{array}{ll}
    \lab{name}=n,\lab{abbrev}=\some{n_a},\lab{inherits}=qs_i,\\
    \lab{clientOf}= qs_c,\lab{explanation}=t,\lab{parts}=ps
    \end{array}\right\}}{\{n \mapsto e_c,n_a \mapsto e_c\}}
}            

\inferrule{
  \forall q \in qs_i, \validinh{e_c}{\qlookup{\env}{q}} \\
  \forall q \in qs_c, \validclient{e_c}{\qlookup{\env}{q}} \\
  \wbx{\env}{t} \\
  \forall p \in ps, \wbx{\env}{p}  \\
}
{
  \wb{\env}{e_c@\lab{Component}\left\{\begin{array}{ll}
    \lab{name}=n,\lab{abbrev}=\none{\nt{name}},\lab{inherits}=qs_i,\\
    \lab{clientOf}= qs_c,\lab{explanation}=t,\lab{parts}=ps
    \end{array}\right\}}{\{n \mapsto e_c\}}
}            

\inferrule{
  \wbx{\env}{t}
}{
  \wbx{\env}{\lab{Constraint}\{\lab{contents}=t\}}
}

\inferrule{
  \wbx{\env}{t}
}{
  \wbx{\env}{\lab{Query}\{\lab{contents}=t\}}
}

\inferrule{
  \wbx{\env}{t}
}{
  \wbx{\env}{\lab{Command}\{\lab{contents}=t\}}
}


\inferrule{
  \qlookup{\env_0}{q} = e \\
  \forall q \in qs_c, \validclient{e}{\qlookup{\env}{q}} \\
}
{
  \wb{\env}{e_c@\lab{ComponentImport}\{\lab{name}=q,\lab{abbrev}=\some{n_a},\lab{clientOf}= qs_c\}\}}{\{n_a \mapsto e_c\}}
}

\inferrule{
  \qlookup{\env_0}{q} = e \\
  \forall q \in qs_c, \validclient{e}{\qlookup{\env}{q}} \\
}
{
  \wb{\env}{e_c@\lab{ComponentImport}\{\lab{name}=q,\lab{abbrev}=\none{\nt{name}},\lab{clientOf}= qs_c\}\}}{\{\listlast{q} \mapsto e_c\}}
}


\inferrule{
  \wb{\env}{is}{\env'}
}
{
  \wb{\env}{e@\lab{Events}\{\lab{name}=n,\lab{contents}=is\}}{\{n\mapsto e\}\uplus \env'}
}


\inferrule{
  \wb{\env}{is}{\env'}
}
{
  \wb{\env}{e@\lab{Scenarios}\{\lab{name}=n,\lab{contents}=is\}}{\{n\mapsto e\} \uplus \env'}
}

\inferrule{
  \wb{\env}{is}{\env'}
}
{
  \wb{\env}{e@\lab{Requirements}\{\lab{name}=n,\lab{contents}=is\}}{\{n\mapsto e\} \uplus \env'}
}

\inferrule{
}{  \wb{\env}{\nillist{\nt{item}}}{\{\}}
}

\inferrule{
  \wb{\env}{i}{\env''} \\
  \wb{\env}{is}{\env'''}
}{
  \wb{\env}{\conslist{i}{is}}{\env'' \uplus \env'''}
}

\inferrule{
  \wbx{\env}{t}
}{
  \wb{\env}{i@\lab{Item}\{\lab{id}=n,\lab{value}=t\}}{\{n\mapsto i\}}
}
  
\inferrule{
  \qlookup{\env}{q} = e \\
  \forall q \in qs_i, \validinh{e}{\qlookup{\env}{q}} \\
  \forall q \in qs_c, \validclient{e}{\qlookup{\env}{q}}
}{
  \wb{\env}{\lab{Relation}\{\lab{name}=q,\lab{inherits}=qs_i,\lab{clientOf}=qs_c\}}{\{\}}
}

\inferrule{
  \forall n_t \in \qnames{t}, \exists e, \env(n_t) = e
}{
  \wbx{\env}{t}
}

\inferrule{
}{\validcontains{\lab{System}\{\ldots\}}{\lab{(Subsystem|SubsystemImport|Relation)}\{\ldots\}}
}

\inferrule{
}{\validcontains{\lab{Subsystem}\{\ldots\}}{\lab{(Subsystem|SubsystemImport|Component|ComponentImport)}\{\ldots\}}
}

\inferrule{
}{\validcontains{\lab{Subsystem}\{\ldots\}}{\lab{(Scenarios|Requirements|Events|Relation)}\{\ldots\}}
}


\inferrule{
  (e_1,e_2) \in I_0
}{\validinh{e_1@\lab{Component}\{\ldots\}}{e_2@\lab{Component}\{\ldots\}}
}

\inferrule{
}{\validclient{\lab{(Subsystem|SubsystemImport)}\{\ldots\}}{\lab{(Subsystem|SubsystemImport|Component|ComponentImport)}\{\ldots\}}
}

\inferrule{
}{\validclient{\lab{(Component|ComponentImport)}\{\ldots\}}{\lab{(Subsystem|SubsystemImport|Component|ComponentImport)}\{\ldots\}}
}

\end{mathparpagebreakable}

\end{document}

\subsection{Implicit Container Relations}

\newcommand{\icmap}{\Sigma}

\newcommand{\bldicmap}[3]{{#1} \vdash {#2} \Rightarrow {#3}}

\newcommand{\reachable}[2]{\texttt{reachable}_{#1}({#2})}
% Reachability should be reflexive.
\newcommand{\nocycles}[1]{\texttt{nocycles}({#1})}
\newcommand{\validic}[2]{\texttt{valid-impl-contains}({#1},{#2})}

These judgments check that the implicit container relations clauses are well formed
and produce a map $\icmap: \nt{uid} \rightarrow \nt{uid}$ mapping each
implicitly contained element to its immediately containing element.

\begin{mathparpagebreakable}
  \inferrule{
  }{\bldicmap{\uidmap}{\nillist{\nt{relation}}}{\emptym}
  }
    
\inferrule{
  \bldicmap{\uidmap}{rs}{\icmap} \\
%  u_c \neq u_p \\  not needed if reachability is reflexive
  \validic{\uidmap(u_c)}{\uidmap(u_p)} \\
  u_c \not\in \reachable{\icmap}{u_p} \\
  \icmap' = \icmap \uplus \{ u_c \mapsto u_p \} \\
}{
  \bldicmap{\uidmap}{\conslist{\lab{ImplicitContains}\{\lab{child}=u_c,\lab{parent}=u_p\}}{rs}}{\icmap'}
}

\inferrule{
  \bldicmap{\uidmap}{rs}{\icmap} \\
}{
  \bldicmap{\uidmap}{\conslist{\lab{Contains}\{\ldots\}}{rs}}{\icmap}
}

\inferrule{
  \bldicmap{\uidmap}{rs}{\icmap} \\
}{
  \bldicmap{\uidmap}{\conslist{\lab{Inherit}\{\ldots\}}{rs}}{\icmap}
}

\inferrule{
  \bldicmap{\uidmap}{rs}{\icmap} \\
}{
  \bldicmap{\uidmap}{\conslist{\lab{Client}\{\ldots\}}{rs}}{\icmap}
}


\inferrule{
}{\validic{\lab{Subsystem}\{\ldots\}}{\lab{System}\{\ldots\}}
}

\inferrule{
}{\validic{\lab{(Subsystem|Component|Scenarios|Requirements|Events)}\{\ldots\}}{\lab{Subsystem}\{\ldots\}}
}

\\

\inferrule{
}{x \in \reachable{M}{x}}

\inferrule{
  M(z) = x \\
  z \in \reachable{M}{y}
}{x \in \reachable{M}{y}
}


\end{mathparpagebreakable}

\subsection{Binding Contexts}

\newcommand{\con}{\Gamma}
\newcommand{\iccon}{\Delta}

These judgements check that the set of names 
defined by each implicit container has no duplicates,
and so uniquely defines a context $\con: \nt{name} \rightarrow \nt{uid}$.
The judgements 
produce a map $\iccon: \nt{uid} \rightarrow \con$
from each implicit container to the context it defines.

\newcommand{\bldcon}[2]{\vdash {#1} \Rightarrow {#2}}
\newcommand{\bldiccon}[3]{{#1} \vdash {#2} \Rightarrow {#3}}

\begin{mathparpagebreakable}

\inferrule{
  n_1 \neq n_2
}{
  \bldcon{\lab{(System|Subsystem|Component)}\{\lab{uid}=u,\lab{name}=n_1,\lab{abbrev}=\some{n_2},\ldots\}}{\{ n_1 \mapsto u, n_2 \mapsto u \}}
}

\inferrule{
}{
  \bldcon{\lab{(System|Subsystem|Component)}\{\lab{uid}=u,\lab{name}=n,\lab{abbrev}=\none{\nt{name}},\ldots\}}{\{ n \mapsto u\}}
}

\inferrule{
}{
  \bldcon{\lab{(Events|Scenarios|Requirements)}\{\lab{uid}=u,\lab{name}=n,\ldots\}}{\{ n \mapsto u\}}
}


\inferrule{
    \bldiccon{\icmap}{es}{\iccon} \\
    u_p = \icmap(\proj{e}{\lab{uid}}) \\
    \con_0 = \left\{{\begin{array}{ll}
                       \iccon(u_p)&\text{if}~u_p \in Dom(\iccon)\\
                       \emptym & \text{otherwise}
                   \end{array}}\right. \\\\
    \vdash e : \con \\
    \iccon' = \iccon[u_p \mapsto (\con_0 \uplus \con)] \\
}{\bldiccon{\icmap}{\conslist{e}{es}}{\iccon'}
}

\inferrule{
  \bldiccon{\icmap}{es}{\iccon}
}{
  \bldiccon{\icmap}{\lab{Source}\{\lab{elems}=es,\ldots\}}{\iccon}
}
  

\end{mathparpagebreakable}

\subsection{Explicit Relations}

\newcommand{\inhmap}{I}
\newcommand{\xcmap}{K}
\newcommand{\clmap}{\Pi}
\newcommand{\bldermap}[6]{{#1},{#2} \vdash {#3} \Rightarrow ({#4},{#5},{#6})}

\newcommand{\validinh}[2]{\texttt{valid-inherit}({#1},{#2})}
\newcommand{\validcontains}[2]{\texttt{valid-contains}({#1},{#2})}
\newcommand{\validclient}[2]{\texttt{valid-client}({#1},{#2})}

These judgements check that the explicit relations are well-formed,
and build an inheritance map
$\inhmap: \nt{uid} \rightarrow \nt{uid}$ from sub-things to super-things,
an explicit containment map
$\xcmap: \nt{uid} \rightarrow \nt{uid}$ from children to parents,
and a client relation
$\clmap: \nt{uid} \times \nt{uid}$ between clients and providers.

\emph{
  TBD:
  \begin{itemize}
  \item Unclear where we will get the contexts $\con$.
  \item Need to fill in the validity definitions.
  \item Treating $\inhmap$ as a map implies single inheritance.
  \item Treating $\xcmap$ as a map implies a thing is contained in at most one other thing.
  \item Need to consider interaction of implicit and explicit relation maps, in particular to find cycles across the combination of the two.
  \item Calculation of $\clmap$ doesn't check for duplicate declarations of a given relation; should it?
  \end{itemize}
}


\begin{mathparpagebreakable}

\inferrule{
}{
  \bldermap{\uidmap}{\con}{\nillist{\nt{relation}}}{\emptym}{\emptym}{\emptym}
}

\inferrule{
  \bldermap{\uidmap}{\con}{rs}{\inhmap}{\xcmap}{\clmap} \\
  u_{sub} = \con(n_{sub}) \\
  u_{super} = \con(n_{super}) \\
  u_{sub} \not \in \reachable{\inhmap}{u_{super}} \\
  \validinh{\uidmap(u_{sub})}{\uidmap(u_{super})} \\
  \inhmap' = \inhmap \uplus \{ u_{sub} \mapsto u_{super} \}
}{
  \bldermap{\uidmap}{\con}{\conslist{\lab{Inherit}\{\lab{sub}=n_{sub},\lab{super}=n_{super}\}}{rs}}{\inhmap'}{\xcmap}{\clmap}
}


\inferrule{
  \bldermap{\uidmap}{\con}{rs}{\inhmap}{\xcmap}{\clmap} \\
  u_c = \con(n_c) \\
  u_p = \con(n_p) \\
  u_c \not \in \reachable{\xcmap}{u_p} \\
  \validcontains{\uidmap(u_c)}{\uidmap(u_p)} \\
  \xcmap' = \xcmap \uplus \{ u_c \mapsto u_p \}
}{
  \bldermap{\uidmap}{\con}{\conslist{\lab{Contains}\{\lab{child}=n_c,\lab{parent}=n_p\}}{rs}}{\inhmap}{\xcmap'}{\clmap}
}

          \\
          
\inferrule{
  \bldermap{\uidmap}{\con}{rs}{\inhmap}{\xcmap}{\clmap} \\
  u_c = \con(n_c) \\
  u_p = \con(n_p) \\
  \validclient{\uidmap(u_c)}{\uidmap(u_p)} \\
  \clmap' = \clmap \cup \{ (u_c,u_p) \}
}{
  \bldermap{\uidmap}{\con}{\conslist{\lab{Client}\{\lab{client}=n_c,\lab{provider}=n_p\}}{rs}}{\inhmap}{\xcmap}{\clmap'}
}


\\

\inferrule{
  \bldermap{\uidmap}{\con}{rs}{\inhmap}{\xcmap}{\clmap} \\
}{
  \bldermap{\uidmap}{\con}{\conslist{\lab{ImplicitContains}\{\ldots\}}{rs}}{\inhmap}{\xcmap}{\clmap}
}
          
\end{mathparpagebreakable}

\subsection{Overall Well-Formedness}

\emph{
  TBD:
  \begin{itemize}
  \item Use of $\con = \iccon(u)$ is bogus; need to fix once we understand desired scoping rules better.
  \item Need to describe interaction of $\icmap$ and $\xcmap$ and perhaps build their union.
  \end{itemize}
}
            
\begin{mathparpagebreakable}

\inferrule{
  \blduidmap{s}{\uidmap} \\
  \bldicmap{\uidmap}{rs}{\icmap} \\
  \bldiccon{\icmap}{s}{\iccon} \\
  \con = \iccon(u) \\
  \bldermap{\uidmap}{\con}{rs}{\inhmap}{\xcmap}{\clmap} \\\\
  \text{$\exists$ at most one $e \in es$ s.t. $e = \lab{System}\{\ldots\}$} \\
}{
  \vdash s@\lab{Source}\{\lab{uid}=u,\lab{elems}=es,\lab{relations}=rs\} \Rightarrow (\uidmap,\icmap,\iccon,\inhmap,\xcmap,\clmap)
}
\end{mathparpagebreakable}




