\documentclass{article}
\usepackage{fullpage}
\usepackage[T1]{fontenc}
\usepackage{longtable}
\usepackage{textcomp}
\usepackage{array}
\usepackage{todonotes}
\usepackage{mathtools}

% Grammar definition helpers
\newenvironment{grammar}{%
  \global\def\grammarVertAdjustment{0mm}
  \newcommand{\gline}{\hline \\[-0.5em]}
  \newcommand{\gcont}{\\ & &}
  \newcommand{\gor}{\ensuremath{\mid}}
  \newcommand{\grule}[4][0mm]{##2 & ::= & ##3 & \mbox{\scriptsize $\textit{##4}$} \\[##1] }
  \newcommand{\gruleplain}[3][0mm]{##2 & \hphantom{::=} & & \mbox{\scriptsize $\textit{##3}$} \\[##1] }
  \small
  \begin{center}
  \begin{longtable}{>{$}r<{$} >{$}c<{$} >{$}l<{$} >{$}r<{$}}
}{%
  \end{longtable}
  \end{center}
}

\newcommand{\txss}[1]{\textsuperscript{#1}}

\newcommand{\glist}[1]{\ensuremath{\mathop{\operatorname{\{} #1 \operatorname{\}}}}}

\newcommand{\glistnonempty}[1]{\ensuremath{\mathop{\operatorname{\{} #1 \operatorname{\}}^+}}}

\newcommand{\gopt}[1]{\ensuremath{\mathop{\operatorname{[} #1 \operatorname{]}}}}

\newcommand{\gntn}[1]{\ensuremath{\mathop{\textrm{#1}}}\,}

\newcommand{\gtkw}[1]{\ensuremath{\mathop{\texttt{"#1"}}}\,}
\newcommand{\gtkc}[1]{\ensuremath{\mathop{\texttt{\textquotesingle #1\textquotesingle}}}\,}

\newcommand{\gtregexfull}[2][]{\ensuremath{\mathop{?\, \textrm{\scriptsize /#2/#1} \,?}}}
\newcommand{\gtregex}[2][]{\ensuremath{\textrm{\scriptsize /#2/#1}}}

\newcommand{\gseplist}[2]{#2 \glist{#1 #2}\!\txss{*}}

% Terminals and non-terminals
\newcommand{\gnnl}{\gntn{eol}}
\newcommand{\gneof}{\gntn{eof}}
\newcommand{\gnkeyword}{\gntn{keyword}}
\newcommand{\gnnlkeyword}{\gntn{nl-keyword}}
\newcommand{\gnnlsystemkeyword}{\gntn{nl-sys-keyword}}
\newcommand{\gnnlsubsystemkeyword}{\gntn{nl-subsys-keyword}}
\newcommand{\gnblockend}{\gntn{block-end}}
\newcommand{\gnident}{\gntn{identifier}}
\newcommand{\gnsentence}{\gntn{sentence}}
\newcommand{\gnsentencelist}{\gntn{sentence-list}}
\newcommand{\gnparagraph}{\gntn{paragraph}}
\newcommand{\gnstring}{\gntn{string}}
\newcommand{\gnstringlist}{\gntn{string-list}}

\newcommand{\gncommentchar}{\gtkw{//}}
\newcommand{\gncomment}{\gntn{\ensuremath{\mathcal{C}}}}
\newcommand{\gnlinecomments}{\gntn{\ensuremath{\mathcal{LC}}}}

\newcommand{\gnlando}{\gntn{lando-source}}
\newcommand{\gnspecelem}{\gntn{spec-element}}
\newcommand{\gnclassdict}{\gntn{class-dictionary}}

\newcommand{\gnsystem}{\gntn{system}}
\newcommand{\gnsubsystem}{\gntn{subsystem}}

\newcommand{\gncomponent}{\gntn{component}}
\newcommand{\gncomponentpart}{\gntn{component-part}}
\newcommand{\gnconstraint}{\gntn{constraint}}
\newcommand{\gnquery}{\gntn{query}}
\newcommand{\gncommand}{\gntn{command}}

\newcommand{\gnevent}{\gntn{event}}
\newcommand{\gnscenario}{\gntn{scenario}}
\newcommand{\gnrequirements}{\gntn{requirement}}

\newcommand{\gnname}{\gntn{name}}
\newcommand{\gnnamephrase}{\gntn{name-phrase}}
\newcommand{\gnnameabbrev}{\gntn{name-abbreviation}}
\newcommand{\gnnamephraserel}{\gntn{name-phrase-rel}}
\newcommand{\gnnamelist}{\gntn{name-list}}
\newcommand{\gnexplanation}{\gntn{explanation}}
\newcommand{\gnnamephraselist}{\gntn{name-phrase-list}}
\newcommand{\gnclassdictentry}{\gntn{dictionary-entry}}


\newcommand{\gnevententry}{\gntn{event-entry}}
\newcommand{\gneventdirection}{\gntn{event-direction}}

\newcommand{\gnscenarioentry}{\gntn{scenario-entry}}

\newcommand{\gncreationentry}{\gntn{creation-entry}}

\newcommand{\gnrequirementsentry}{\gntn{req-entry}}

\newcommand{\gnindexing}{\gntn{indexing}}
\newcommand{\gnindexentry}{\gntn{index-entry}}
\newcommand{\gnindexkey}{\gntn{index-key}}
\newcommand{\gnindexvallist}{\gntn{index-val-list}}
\newcommand{\gnindexval}{\gntn{index-val}}

\newcommand{\gnrelkeyword}{\gntn{rel-keyword}}

\newcommand{\gnrelation}{\gntn{relation}}


\newcommand{\gnn}{\ensuremath{\mathcal{N}}}
\newcommand{\gnl}{\ensuremath{\mathcal{L}}}
\newcommand{\gnb}{\ensuremath{\mathcal{B}}}
\newcommand{\gnc}{\ensuremath{\mathcal{C}}}
\newcommand{\gns}{\ensuremath{\mathcal{S}}}

%\usepackage{cmbright}
%\usepackage[varg]{txfonts}
%\usepackage{pxfonts}

\begin{document}

\title{Lando System Specification Language v2 Grammar}
\maketitle

\section{Notation}

The grammar for the Lando System Specification Sublanguage is written in the EBNF notation. The main elements of the notation that we utilize are:
\begin{itemize}
  \item Keyword terminals are represented in $\gtkw{typewriter}$ font. 
  \item Single-character terminals appear in single quotes, e.g. $\gtkc{!}$ or $\gtkc{\gtesc{n}}$.
  \item More complex terminals are represented using standard regular expressions written as $\gtregex{regexp}$, which may include escaped control characters such as $\gtesc{n}$. Regular expressions may use character classes (with negation). Note that the pattern $\gtregex{/}$ matches the literal character forward slash.
  \item An optional $\gntn{thing}$ is written as $\gopt{\gntn{thing}}$.
  \item {Zero or more repetitions of $\gntn{thing}$ are written as $\glist{\gntn{thing}}$}.
  \item One or more repetitions of $\gntn{thing}$ are written as $\glistnonempty{\gntn{thing}}$.
  \item If $S$ and $T$ are non-terminals or regular expressions, $S - T$ denotes the set of strings matching S but not T.
  \item $\gntn{EOF}$ represents end-of-file; it can be thought of as a special terminal that is not consumed, but rather is replicated as often as necessary to match the grammar when parsing.
  \item Whitespace and line separators are significant throughout. Note that the line separator terminal $\gnn$ consumes any whitespace immediately following the actual end-of-line character(s). Two consecutive line separators (i.e. matching $\gnn\gnn$) are always lexed instead as a single blank line terminal $\gnb$;
    this distinction is necessary to resolve certain ambiguities in the grammar for paragraphs.
    The general line separator $\gnl$ is the union of $\gnn$ and $\gnb$.
 \item In cases of ambiguity, productions involving repetition behave ``greedily,'' i.e. they match as much of the input as possible.
    For example, because the Sentence Body production uses $\glist{}$,  the string ``\gtkw{abc def ghi}'' parses as a single sentence
    containing three words, rather than as three sentences of one word each.
\end{itemize}

\vfill\eject

\subsection{Context-Free Grammar}

\begin{grammar}
  \grule{\gntn{landoSource}}{
    \glist{\gnl} 
    \glist{\gntn{body}} \glist{\gnl} 
    \glist{\gnc\gnl} \glist{\gnl} 
    \gntn{EOF}
  }{Lando source}

  \grule{\gntn{specElement}}{
    \gntn{system} \gor \gntn{subsystem} \gor \gntn{subsystemImport} \gor \gntn{component}  \gor \gntn{componentImport} \gcont
    \gor \gntn{events} \gor \gntn{scenarios} \gor \gntn{requirements} \gor \gntn{relation}
  }{Top-level element}

  \gline

  \grule{\gntn{system}}{
    \glist{\gnc\gnl}
    \gtkw{system} \gntn{name} \gopt{\gntn{abbrev}} \gopt{\gnc} \glistnonempty{\gnl} \gcont
    \gntn{explanation}  \glist{\gnc \gnl}
    \gopt{\gntn{indexing} \gntn{blockend}} \gcont
    \gopt{\gtkw{contains} \glist{\gnl} \gntn{body} \gtkw{end} \gopt{\gnc} \gntn{blockend}}
  }{System}

  \grule{\gntn{body}}{
    \glist{\gntn{specElement}} 
  }{Source/(Sub)system body}

  \grule{\gntn{subsystem}}{
    \glist{\gnc\gnl}
    \gtkw{subsystem} \gntn{name} \gopt{\gntn{abbrev}} 
    \glist{clientClause} \gopt{\gnc} \glistnonempty{\gnl} \gcont
    \gntn{explanation}  \glist{\gnc \gnl}
    \gopt{\gntn{indexing} \gntn{blockend}} \gcont
    \gopt{\gtkw{contains} \glist{\gnl} \gntn{body} \gtkw{end} \gopt{\gnc} \gntn{blockend}}
  }{Subsystem}
  
  \grule{\gntn{subystemImport}}{
    \glist{\gnc\gnl}
    \gtkw{import} \glistnonempty{\gns} \gtkw{subsystem} \gntn{name} \gopt{\gntn{abbrev}}
    \glist{clientClause} \gopt{\gnc} \gntn{blockend}
  }{Subsystem import}
  
  \grule{\gntn{explanation}}{
    \gntn{paragraph}
  }{Explanation}

  \grule{\gntn{abbrev}}{
    \glist{\gns} \gtkc{(} \glist{\gns} \gntn{nameWord} \glist{\gns} \gtkc{)} \glist{\gns}
  }{Abbreviation}

  \grule{\gntn{inheritClause}}{
    \gtkw{inherit} \gntn{qname} \glist{\gnl} \glist{\gtkc{,} \glist{\gnl} \gntn{qname} \glist{\gnl}}
  }{Inherit clause}

  \grule{\gntn{clientClause}}{
    \gtkw{client} \gntn{qname} \glist{\gnl} \glist{\gtkc{,} \glist{\gnl} \gntn{qname} \glist{\gnl}}
  }{Client clause}
    
  \grule{\gntn{indexing}}{
    \gtkw{indexing} \glist{\gns} \glist{\glistnonempty{\gnl} \gntn{indexEntry}}
  }{Indexing}

  \grule{\gntn{indexEntry}}{
    \gntn{indexValue} \gtkc{:} \gntn{indexValuePart} \glist{\glistnonempty{\gnl} \gntn{indexValuePart}}
  }{Index Entry}

  \grule{\gntn{indexValuePart}}{
    \gntn{indexValue} \glist{\gnl} \gopt{\gnc}
  }{Index Value Part}

  \gline

  \grule{\gntn{component}}{
    \glist{\gnc\gnl}
    \gtkw{component} \gntn{name} \gopt{\gntn{abbrev}} 
    \glist{inheritClause \gor clientClause} \gopt{\gnc} \glistnonempty{\gnl} \gcont
    \gntn{explanation}  
    \glist{\gntn{componentFeature} \gntn{blockend}}
  }{Component definition}

  \grule{\gntn{componentImport}}{
    \glist{\gnc\gnl}
    \gtkw{import} \glistnonempty{\gns} \gtkw{component} \gntn{name} \gopt{\gntn{abbrev}}
    \glist{clientClause} \gopt{\gnc}
    \gntn{blockend}
  }{Component import}
    
  \grule{\gntn{componentFeature}}{
    \gntn{constraint} \gor \gntn{command} \gor \gntn{query}
  }{Component Feature}

  \grule{\gntn{constraint}}{
    \glist{\gnc\gnl} \gntn{sentBody} \gtkc{.} \gopt{\gntn{wordSep}} \gopt{\gnc}
  }{Constraint Feature}
    
  \grule{\gntn{query}}{
    \glist{\gnc\gnl} \gntn{sentBody} \gtkc{?} \gopt{\gntn{wordSep}} \gopt{\gnc}
  }{Query Feature}

  \grule{\gntn{command}}{
    \glist{\gnc\gnl} \gntn{sentBody} \gtkc{!} \gopt{\gntn{wordSep}} \gopt{\gnc}
  }{Command Feature}

  \gline

  \grule{\gntn{events}}{
    \glist{\gnc\gnl}
    \gtkw{events}
    \gntn{name} \gopt{\gnc}
    \glistnonempty{\gnl} \glist{\gntn{eventEntry}}
  }{Events}


  \grule{\gntn{eventEntry}}{
    \glist{\gnc\gnl}
    \gntn{name} \gopt{\gnc}
    \glistnonempty{\gnl}
    \gntn{paragraph} 
  }{Event Entry}


  
  \grule{\gntn{scenarios}}{
    \glist{\gnc\gnl}
    \gtkw{scenarios}
    \gntn{name} \gopt{\gnc}
    \glistnonempty{\gnl} \glist{\gntn{scenarioEntry}}
  }{Scenarios}


  \grule{\gntn{scenarioEntry}}{
    \glist{\gnc\gnl}
    \gntn{name} \gopt{\gnc}
    \glistnonempty{\gnl}
    \gntn{paragraph} 
  }{Scenario Entry}


  \grule{\gntn{requirements}}{
    \glist{\gnc\gnl}
    \gtkw{requirements}
    \gntn{name} \gopt{\gnc}
    \glistnonempty{\gnl} \glist{\gntn{requirementEntry}}
  }{Requirements}


  \grule{\gntn{requirementEntry}}{
    \glist{\gnc\gnl}
    \gntn{name} \gopt{\gnc}
    \glistnonempty{\gnl}
    \gntn{paragraph}
  }{Requirement Entry}

  \gline

  \grule{\gntn{relation}}{
    \glist{\gnc\gnl}
    \gtkw{relation} \gntn{qname} \glistnonempty{\gntn{inheritClause} \gor \gntn{clientClause}}
    \gopt{\gnc}
    \gntn{blockend}
  }{Relation Declaration}


  \gline
  
  \grule{\gntn{qname}}{
    \gntn{name} \gor \gntn{qname} \gtkc{:} \gntn{name}
  }{Name}

  \grule{\gntn{name}}{
    \glist{\gns} \gntn{nameWord} \glist{\glistnonempty{\gns} \gntn{nameWord}} \glist{\gns}
  }{Name facet}

  \grule{\gntn{nameWord}}{
    \gtregex{[\gtneg\gtesc{r}\gtesc{n} \gtesc{t}():]*[\gtneg\gtesc{r}\gtesc{n} \gtesc{t}():,.!?]} - \gtregex{//.*} - \gntn{nonName}
  }{Name word}

  \grule{\gntn{nonName}}{
    \gtkw{client} \gor \gtkw{inherit} % \gor \ldots ??
  }{Keywords}

  \gline

  \grule{\gntn{indexValue}}{
    \glist{\gns} \gntn{indexValueWord} \glist{\glistnonempty{\gns} \gntn{indexValueWord}} \glist{\gns}
  }{Index value}

  \grule{\gntn{indexValueWord}}{
    \gtregex{[\gtneg\gtesc{r}\gtesc{n} \gtesc{t}]*[\gtneg\gtesc{r}\gtesc{n} \gtesc{t}:]} - \gtregex{//.*} 
  }{Index value word}

  \gline

  \grule{\gntn{paragraph}}{
    \glistnonempty{\gntn{sentence}} \gntn{parend}
  }{Paragraph}

  \grule{\gntn{parend}}{
    \gopt{\gnc} (\gnb \glist{\gnl} \gor \gntn{EOF})
  }{Paragraph end}

  \grule{\gntn{sentence}}{
    \gntn{sentBody} \gopt{\gntn{sentTerm}} \gopt{\gntn{wordSep}} 
  }{Sentence}
    
  \grule{\gntn{sentBody}}{
    \gntn{sentWord} \glist{\gntn{wordSep} \gntn{sentWord}} \gopt{\gntn{wordSep}}
  }{Sentence Body}

  \grule{\gntn{wordSep}}{
    \glistnonempty{\gns} \gor \glist{\gns} \gopt{\gnc} \gnl 
  }{Word Separator}

  \grule{\gntn{sentWord}}{
    \gtregex{[\gtneg\gtesc{r}\gtesc{n} \gtesc{t}]*[\gtneg\gtesc{r}\gtesc{n} \gtesc{t}.!?]} - \gtregex{//.*} 
  }{Sentence word}

  \grule{\gntn{sentTerm}}{
    \gtkc{.} \gor \gtkc{!} \gor \gtkc{?}
  }{Sentence terminator}

  \grule{\gntn{blockend}}{
    \glistnonempty{\gnl} \gor \gntn{EOF}
  }{Block end}  

  \gline

  \grule{\gns}{
    \gtregex{[ \gtesc{t}]}
  }{Whitespace character}

  \grule{\gnn}{
    \gopt{\gtkc{\gtesc{r}}} \gtkc{\gtesc{n}} \gopt{\gns} \gor \gtkc{\gtesc{r}} \gopt{\gns} \gor \gntn{EOF}
  }{Single line separator}

  \grule{\gnb}{
    \gnn\gnn
  }{Blank line}  

  \grule{\gnl}{
    \gnn \gor \gnb
  }{General line seperator}

  \grule{\gnc}{
    \gtregex{//[\gtneg\gtesc{r}\gtesc{n}]*}
  }{Comment}

\end{grammar}

\section{Context-Sensitive Considerations}

The following extra-grammatical rules are applied during or after parsing the concrete grammar above.

\begin{itemize}
\item 
  If a \gntn{system} or \gntn{subsystem} element does not have an explicit body, then it has
  an implicit body that begins immediately after the element and extends until just
  before the next \gtkw{system}, \gtkw{subsystem} or unmatched \gtkw{end}.
\item
  \gntn{subystemImport} an only appear in an (explicit or implicit) \gntn{system} or \gntn{subsystem}
  body.  \gntn{component} and \gntn{componentImport} can only appear in an (explicit or implicit)
  \gntn{subsystem} body.
\item
  A \gntn{sentence} that does not end with a \gtkc{.},\gtkc{!}, or \gtkc{?} symbol
  (possibly followed by white space characters or line separators) induces a warning message.
\item
  Note that because of the ``greedy'' resolution of ambiguity, this warning can only occur at the end of a paragraph.
\end{itemize}

\end{document}

