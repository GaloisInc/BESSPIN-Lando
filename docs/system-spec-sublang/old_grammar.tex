\documentclass{article}

\input{old_macros}

\usepackage{cmbright}
%\usepackage[varg]{txfonts}
%\usepackage{pxfonts}

\begin{document}

\section{Notation}

The grammar for the Lando System Specification Sublanguage is written in the EBNF notation. The main elements of the notation that we utilize are:
\begin{itemize}
  \item Terminals are represented with double or single quotes; e.g. $\gtkw{explanation}$.
  \item Optional bits are represented with squared brackets; e.g. $\gopt{\gtkw{explanation} \gnparagraph}$.
  \item Repetition is represented with curly braces; e.g. $\glist{\gnident}$
  \item We use a slightly enhanced notation - $\glistnonempty{a}$ to indicate \emph{non-zero} repetitions. This is simply equivalent to: $a \glist{a}$.
  \item For defining terminals, we would like to use the EBNF special form to declare an extended regular expression - for example: $\gtregexfull{\textbackslash w\\+}$. However since this is rather verbose, we will simply use $\gtregex{\textrm{\textbackslash w\\+}}$ for convenience.
  \item Blocks in the language are typically delimited by keywords indicating the start of \emph{another} block. In defining the grammar this translates to lookaheads: (i.e. peeking at incoming tokens without consuming them). We use the perl regular expression format to indicate this. E.g. $\gtregex{?= (\gnnl \gtkw{system})}$.
  \item Many constructs in the grammar include significant whitespaces (e.g. $\gnnamephrase$). However note that we only consider \emph{internal} whitespaces to be significant. For example, a index entry ``Email Address : foo <foo@example.com> '' is interpreted as the mapping - ``Email Address'' $\mapsto$ ``foo <foo@example.com>'' and not ``Email Address '' $\mapsto$ ``foo <foo@example.com> ''
\end{itemize}

\section{Grammar}

\begin{grammar}
  \grule{\gnlando}{
    \glist{\gnspecelem}
  }{Lando source}
  
  \grule{\gnspecelem}{
    \gnsystem \gor \gnsubsystem \gor \gncomponent \gor \gnevent \gor \gcont 
    \gnscenario \gor \gnrequirements
  }{Specification Elements}

  \gline

  \grule{\gnsystem}{
    \gnlinecomments \gtkw{system} \gnnamephraserel \gcont
    \; \gopt{\gnrelkeyword \gnnamephrase} \gncomment \gnnl \gcont
    \; \gnexplanation \gnnl \gcont
    \; \gopt{\gtkw{indexing} \gnnl \gnindexing \gnnl} \gcont
    \; \gseplist{\gnnl}{\gnsubsystem} \gcont
    \; \gtregex{(?= \gnnlsystemkeyword | \gneof)} \gnblockend
  }{System}

  \gline

  \grule{\gnsubsystem}{
    \gnlinecomments \gtkw{subsystem} \gnnamephraserel \gopt{\gtkc{(} \gnnameabbrev \gtkc{)}} \gcont
    \; \gopt{\gnrelkeyword \gnnamephrase} \gncomment \gnnl \gcont
    \; \gnexplanation \gnnl \gcont
    \; \gopt{\gtkw{indexing} \gnnl \gnindexing \gnnl} \gcont
    \; \gseplist{\gnnl}{\gncomponent} \gcont
    \; \gtregex{(?= \gnnlsubsystemkeyword | \gneof)} \gnblockend
  }{Cluster}

  \gline

  \grule{\gncomponent}{
    \gnlinecomments \gtkw{component} \gnnamephraserel \gopt{\gtkc{(} \gnnameabbrev \gtkc{)}} \gcont
    \; \gopt{\gnrelkeyword \gnnamephrase} \gncomment \gnnl \gcont
    \; \gnexplanation \gnnl \gcont
    \; \gseplist{\gnnl}{\gncomponentpart} \gcont
    \; \gtregex{(?= \gnnlkeyword | \gneof)} \gnblockend
  }{Class}

  \grule{\gncomponentpart}{
    \gnconstraint | \gncommand | \gnquery
  }{Component Parts}

  \grule{\gnconstraint}{
    \gtregex[m]{[\text{\textasciicircum}.?!]\txss{+}?\,\textbackslash.]}
  }{Constraint}

  \grule{\gnquery}{
    \gtregex[m]{[\text{\textasciicircum}.?!]\txss{+}?\,\textbackslash?]}
  }{Query}

  \grule[0.5em]{\gncommand}{
    \gtregex[m]{[\text{\textasciicircum}.?!]\txss{+}?\,!]}
  }{Command}

  \gline

  \grule{\gnevent}{
    \gnlinecomments \gtkw{events} \gnnamephrase \gncomment \gnnl \gcont
    \; \gseplist{\gnnl}{\gnevententry} \gcont
    \; \gtregex{(?= \gnnlkeyword | \gneof)} \gnblockend
  }{Events}

  \grule{\gnevententry}{
    \gnnamephrase \gnnl \gnsentence
  }{Event Entry}

  \gline

  \grule{\gnscenario}{
    \gnlinecomments \gtkw{scenarios} \gnnamephrase \gncomment \gnnl \gcont
    \; \gseplist{\gnnl}{\gnevententry} \gcont
    \; \gtregex{(?= \gnnlkeyword | \gneof)} \gnblockend
  }{Scenario}

  \grule{\gnscenarioentry}{
    \gnnamephrase \gnnl \gnsentence
  }{Scenario Entry}

  \gline

  \grule{\gnrequirements}{
    \gnlinecomments \gtkw{requirements} \gnnamephrase \gncomment \gnnl \gcont
    \; \gseplist{\gnnl}{\gnrequirementsentry} \gcont
    \; \gtregex{(?= \gnnlkeyword | \gneof)} \gnblockend
  }{Requirements}

  \grule{\gnrequirementsentry}{
    \gnnamephrase \gnnl \gnsentence
  }{Requirements Entry}

  \gline
  
  \grule{\gnrelation}{
    \gnlinecomments \gtkw{relation} \gnnamephraserel \gnrelkeyword \gnnamephrase \gncomment \gcont
    \; \gtregex{(?= \gnnlkeyword | \gneof)} \gnblockend
  }{Relation Declaration}

  \gline
  
  \grule{\gnindexing}{
    \gseplist{\gnnl}{\gnindexentry}
  }{Index List}

  
  \grule{\gnindexentry}{
    \gnindexkey \gtkc{:} \gnindexvallist
  }{Index List}

  \grule{\gnindexkey}{
    \gtregex{[\text{\textasciicircum:}]\txss{+}}
  }{Index Key}

  \grule{\gnindexvallist}{
    \gseplist{\gnnl}{\gnindexval} \gcont
    \; \gtregex{(?= \gneof | \gnnlkeyword | \gnnl \gnindexkey )}
  }{Index Value List}

  \grule[0.5em]{\gnindexval}{
    \gtregex{[\text{\textasciicircum:}]\txss{+}}
  }{Index Value}
  

  \gline

  \grule{\gnnamephrase}{
      \gtregex{\textbackslash{}w[\textbackslash{}w\textbackslash{}s]*\,(?= \gncomment | \gnnl | \gneof)}
  }{Name-Phrase}
  \grule{\gnnamephraserel}{
    \gtregex{\textbackslash{}w[\textbackslash{}w\textbackslash{}s]*\,(?= \gnrelkeyword | \gncomment | \gnnl | \gneof)}
  }{Name-Phrase - Relation}
  \grule{\gnnameabbrev}{
    \gtregex{[A-Z]\txss{+}}
  }{Name-Abbreviation}
  \grule{\gnsentence}{\gtregex[m]{[\text{\textasciicircum}.?!]\txss{+}? [.?!]}}{Sentence}
  \grule{\gnparagraph}{\gnsentence^+ \gtregex{(?= (\gnnl \gnkeyword | \gneof))}}{Paragraph}
  \grule{\gnexplanation}{\gnparagraph}{Explanation}
  \grule{\gnkeyword}{<all keywords>}{All\ Keywords}
  \grule{\gnnlkeyword}{\gnnl \, \gnkeyword}{Keyword on new line}
  \grule{\gnnlsystemkeyword}{\gnnl \, \gtkw{system}}{}
  \grule{\gnnlsubsystemkeyword}{\gnnl \, (\gtkw{subsystem} | \gtkw{system})}{}
  \grule{\gnrelkeyword}{\gtkw{inherit} | \gtkw{client} | \gtkw{contains}}{Relation Keywords}
  \grule{\gnblockend}{\gnnl | \gneof}{Block End}
  \grule{\gnnl}{\gtregex{(\textbackslash r?\textbackslash n) | \textbackslash r}}{New Line}
  \grule{\gnlinecomments}{\glist{\gncomment \gnnl}}{Opt Line Comments}
  \grule{\gncomment}{\gopt{\gncommentchar \gtregex{.*\,(?= (\gnnl | \gneof))}}}{Opt Comment}

\end{grammar}


\end{document}

